\setcounter{page}{1}
\section{Introduction}
\subsection*{The Applicative Project}
During the second year of our engineering degree at ESAIP, we had to do what is called the ``Projet Applicatif''\footnote{Applicative Project}. This project is the occasion for us to do any kind of project we always wanted to do, but never had the opportunity or time to accomplish.\\

As we are both Big DATA students, we decided to opte for a project using machine learning and/or deep learning. Our first idea was to try to simulate an autonomous car in a virtual environment\footnote{CARLA to be exact}, but then, mid October Mr. KERMORVANT offered to participate in a tournament: \textbf{$<$IA/$>$ ~Racing}.\\

\subsection*{Groupe Sigma and $<$IA/$>$ ~Racing}
In 2020, the \textbf{Groupe Sigma} launched an autonmous RC-Car competition. %\cite{SigmaRacing}.
 6 team of 3 membres (5 teams of students and 1 team of Sigma's employees) are going to compete on a circuit specialy made for the occasion. Each round, 2 cars will drive all alone on the circuit, without any kind of intervention from the teams. The goal of this competition is to initiate student to machine learning and electronic.\\
 
You could ask yourself, ``But you are only two, how can you participate in the race ?''. Well during this Applicative Project we are only two, but for the Sigma Race we're going to be helped by \textbf{Freud MAGUENDJI}. He was already in an other project of computer vision when Mr. KERMORVANT offered us to enter the competition. We did the whole machine learning and simulator setup just by ourselves, and Freud is going to help us for the rest of the project as he is an IOT\footnote{Internet Of Things} student.\\

\hfill \\

\begin{figure}[!h]
	\centering
    \def\svgwidth{0.7\columnwidth}
    \input{sigma.pdf_tex}
\end{figure}
\clearpage
