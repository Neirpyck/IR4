\section{Conclusion}

Grâce à M. \textsc{Girou}, qui a posté très tôt sur \textbf{Linkedin} notre recherche de mission, nous avons eu la chance de trouver notre Activ'ESAIP dès mars. Nous avons ainsi eu l'opportunité de nous renseigner sur l'entreprise, de réfléchir à la méthode de travail que nous allions appliquer,\dots\\
\hfill \\
Bien qu'un peu complexe à cerner au début (beaucoup de concepts nouveaux, un environnement de travail différent,\dots), ce projet s'est avéré très enrichissant sur tous les domaines. Panorama Performance a su se montrer très pédagogue sur des concepts comme les flux-tirés ou le lean management, et nous avons été confronté à un monde auquel nous n'étions pas tous familier.\\

L'autonomie a été un des axes principaux de cet activ'ESAIP : l'équipe de Panorama Performance n'est constituée que de deux personnes à plein temps, il leur était donc compliqué de trouver des trous dans leur calendrier pour pouvoir s'occuper de nous. Mis à part la réunion hebdomadaire, nous n'avions peu (ou pas) de retours ! Mais grâce à la méthode agile que nous avons appliquée, nous avons utilisé ces feedback client pour nous améliorer continuellement. Trop hautes au début, les attentes ont été revues à la baisse pour mieux correspondre à la limite que nous avions en terme de temps. Nous avons ainsi réalisé moins de fonctionnalités, mais nous avons pu les réaliser entièrement et sans faire de code “spaghetti"\footnote{Familièrement, code écrit sans suivre de ligne de conduite professionnel (fonctionnel mais illisible)}.\\
\hfill \\
Malgré notre projet très axé autour de la POC, il faut rappeler que Panorama Performance ne s'intéressait pas seulement à l'aspect technique de la solution, mais aussi à sa rentabilité économique. C'est pourquoi, M. \textsc{Carrillo} s'est chargé de rédiger entièrement un rapport de faisabilité pour présenter les concurrents, le business model le plus intéressant, et les coûts entraînés par un tel projet. Mme. \textsc{Oudard} a su se montrer très réactive pour répondre à nos interrogations sur le fonctionnement économique de l'entreprise pour lui fournir tous les documents nécessaires.\\

Réaliser un projet ingénieur d'une telle envergure n'est pas simple, surtout lorsqu'il est réalisé à distance, dans un contexte de crise sanitaire. Nous avons cependant le gros avantage d'être tout trois colocataires, et avons ainsi pu se soutenir les uns les autres lorsque la concentration venait à baisser. Les dailys-meetings, les réunions de fin de sprint et l'utilisation de moyens de communication comme Slack ou Teams nous ont permis de garder un cadre professionnel même entre nous. \\

En ce qui concerne les livrables, l'entreprise s'est montré enthousiaste vis-à-vis de la POC que nous lui avons livré, et a même évoqué la possibilité d'une future collaboration pour poursuivre ce projet. À l'heure de la rédaction de ces lignes, nous avons encore une semaine en entreprise à réaliser, ce sera pour M. \textsc{Carrillo} l'occasion de finir le business model et pour M. \textsc{Barbault} et M. \textsc{Girou}, une semaine de plus à consacrer au développement de l'application.\\

Cette mission aura été pour nous trois une excellente expérience, grâce à la fois à un sujet intéressant, mais aussi grâce à une équipe qui a su nous présenter sa solution de séquenceur et nous faire comprendre les concepts qu'elle renfermait.


