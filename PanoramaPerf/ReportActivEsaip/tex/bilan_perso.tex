\section{Bilans personnels}
\subsection*{Romain}
\addcontentsline{toc}{subsection}{Romain}

Je me suis concentré principalement sur le développement d’une POC durant ce projet Activ’esaip. Comme chaque projet de développement, l’élément clé est la définition des objectifs. Panorama Performance n’avait pas une idée claire de ce qu’ils souhaitaient développer en début de projet. C’est pourquoi il nous a fallu redéfinir les objectifs à la fin de la deuxième semaine. Cette redéfinition, plus réaliste et faisable, nous a permis de nous concentrer sur les fonctionnalités principales et essentielles à Panorama Performance.\\

Le développement a été en lui même un défi de taille. Cette notion de “simplexité”\footnote{La simplexité est l’art de rendre simples, lisibles, compréhensibles les choses complexes. C'est une notion émergente et un domaine d'étude nouveau en systémique, ingénierie et neurosciences.} a guidée ma façon d’aborder le développement. Cependant, les applications en apparence simple abritent en réalité un code assez complexe. Cela aura donc été un challenge qui, selon moi, a su être relevé.
La solution développée est une application web développé sous Flutter, utilisant les services de Firebase pour le backend, notamment Cloud FireStore, la base de donnée NoSQL. \hfill \\

Globalement, ce projet a été très enrichissant, et ce, sur de nombreux aspects. Le plus important selon moi est la pleine considération de nos compétences et de notre réflexion. Panorama Performance nous a intégrés, depuis le début du projet, comme des ingénieurs. Les questions qui nous été posées n’étaient donc pas limitées à l’aspect technique. Dès lors, nous avons pu prendre pleinement conscience de la vision transversale qu’un ingénieur doit avoir. Nous devons être irréprochables techniquement, mais nous devons aussi être en capacité de réaliser des études de marchés, des business models, …

\subsection*{Cyprien}
\addcontentsline{toc}{subsection}{Cyprien}

Après 4 ans d'études à l'ESAIP et plus particulièrement 2 ans de cycle ingénieur, nous avons enfin été confrontés à un projet professionnalisant. Travailler au contact d'une entreprise ne peut pas être appris dans des livres et on ne peut pas prévoir comment réagir une fois jeté dans le monde réel. Je trouve particulièrement enrichissant d'avoir réalisé ce projet avec un ancien élève de l'école, cela prouve la porté du réseau des anciens.
\hfill \\

Tour à tour développeur et Scrum master, j'ai une fois de plus pu constater la force de la méthode de management agile, de l'amélioration et du retour client. Cela m'a conforté dans mon idée de me spécialiser dans le management d'équipe à moyen ou long terme car, j'aime cet aspect de la gestion de projet. Je n'en suis pas à mon premier projet en compagnie de Romain \textsc{Girou} et nous avons déjà nos routines de travail ; nous savons comment fonctionne l'autre et comment nous répartir les tâches pour que le projet avance au plus vite. C'était cependant la première fois que nous décidions de travailler en pair-programming, une technique intéressante si bien maîtrisé, je pense retenter l'expérience lors d'une prochaine mission si l'occasion se présente.
\hfill \\

Je trouve que Panorama Performance nous a accueilli comme les (futurs) ingénieurs que nous sommes, et nous a laissé carte blanche sur la gestion du projet tant que nous respections les dead-lines et les livrables. Ils nous ont même félicité pour notre autonomie et notre professionnalisme.
\hfill \\

Je retiendrais ce projet Activ'ESAIP comme une excellente expérience, qui m'aura initiés aux différents concepts de la performance industrielle, plus particulièrement au séquenceur et à la force de concepts comme les flux tirés ou de la production juste à temps. J'ai également pu perfectionner ma maîtrise de Flutter, et plus particulièrement au “state management" avec la méthodologie BLoC. Cela me sera très utile pour mon stage de cet été où je serais amené une fois de plus à développer en Flutter.
