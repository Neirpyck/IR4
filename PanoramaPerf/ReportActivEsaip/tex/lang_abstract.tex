%%% Résumé en 3 langues du projet

\section{Résumé en trois langues}
\section*{Français}

Durant le dispositif Activ'ESAIP, notre équipe, constituée de Valentin \textsc{Carrillo}, Romain \textsc{Girou} et Cyprien \textsc{Barbault}, a été amené à travailler avec Panorama Performance. Cette TPE est spécialisée dans l'aide à la performance industrielle. La mission qu'elle nous a confiée est la suivante : est-il possible de digitaliser un de leurs outils, le séquenceur, et d'en faire une solution viable tant pour leur client que pour Panorama. Après une semaine de recherche, nous avons décidé de développer une POC en Flutter pour montrer la faisabilité d'un tel projet. En parallèle, nous avons également rédigé un business modèle présentant les différentes options de financement et les coûts associés. En suivant les lignes de conduite de la méthode agile, nous avons pu rendre des livrables réguliers à nos tuteurs et leur délivrer en temps et en heure une version de test. Suite à cette séance d'essai, Panorama Performance nous a annoncé vouloir présenter un MVP à ses plus fidèles clients pour récolter leurs avis et lancer ou non la production.

\section*{English}

During activ'ESAIP, our team, made up of Valentin \textsc{Carrillo}, Romain \textsc{Girou} and Cyprien \textsc{Barbault}, was brought to work with Panorama Performance. This TPE specialises in industrial performance support. The mission she entrusted to us is this: is it possible to digitize one of their tools, the sequencer, and make it a viable solution for both their client and Panorama. After a week of research, we decided to develop a flutter POC to show the feasibility of such a project. At the same time, we have also drafted a business model presenting the different financing options and associated costs. By following the agile method, we were able to return regular deliverables to our tutors and deliver them a test version on time. Following this test session, Panorama Performance announced that it wanted to present an MVP to its most loyal customers to collect their opinions and launch or not the production.

\section*{Español}

Durante el dispositivo activ'ESAIP, nuestro equipo, compuesto por Valentin \textsc{Carrillo}, Romain \textsc{Girou} y Cyprien \textsc{Barbault}, trabajó con Panorama Performance. Esta TPE se especializa en la ayuda al rendimiento industrial. La misión que nos ha confiado es la siguiente: es posible digitalizar una de sus herramientas, el secuenciador, y hacer de ella una solución viable tanto para su cliente como para Panorama. Después de una semana de investigación, decidimos desarrollar una POC en Flutter para demostrar la viabilidad de tal proyecto. Paralelamente, también hemos elaborado un modelo de negocio que presenta las diferentes opciones de financiación y los costes asociados. Siguiendo las líneas del método ágil, hemos podido entregar entregas regulares a nuestros tutores y entregarles a tiempo una versión de prueba. Tras esta sesión de prueba, Panorama Performance nos ha anunciado que quiere presentar un MVP a sus clientes más fieles para recoger sus opiniones y lanzar o no la producción.
