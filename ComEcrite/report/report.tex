\documentclass[12pt]{lettre}
\usepackage[utf8]{inputenc}
\usepackage[T1]{fontenc}
\usepackage[french]{babel}
\usepackage{tikz}

\makeatletter
   \newcommand*{\NoRule}{\renewcommand*{\rule@length}{0}}
\makeatother


\newcommand{\slice}[4]{
  \pgfmathparse{0.5*#1+0.5*#2}
  \let\midangle\pgfmathresult

  % slice
  \draw[thick,fill=black!10] (0,0) -- (#1:1) arc (#1:#2:1) -- cycle;

  % outer label
  \node[label=#4] at (\midangle:1.5) {};

  % inner label
  \pgfmathparse{min((#2-#1-10)/110*(-0.3),0)}
  \let\temp\pgfmathresult
  \pgfmathparse{max(\temp,-0.5) + 0.8}
  \let\innerpos\pgfmathresult
  \node at (\midangle:\innerpos) {#3};
}

\begin{document}
\begin{letter}{Responsable commercial\\ Monsieur DUBOIS\\}
	\name{Assistant Responsable commercial\\Monsieur BARBAULT~~~~~~~~~~~~~~~~}
	\address{Agence de voyage HAVAS}
	\lieu{Angers}    
	\nofax
	\date{le 3 f\'evrier 2020}
    \notelephone
\NoRule

\conc{Probl\`e{}me dans l'addressage des publipostage}
\opening{}

Suite \`a une action de mailing, le service commercial de notre agence a remarqu\'e un taux de rendement tr\`es faible. De plus de nombreux courriers, lui ont \'et\'es retourn\'es avec la mention "N'habite pas \`a l'addresse".\\

\textbf{I. D\'emarche adopt\'ee}

Pour comprendre ces r\'esultats, nous avons r\'ealis\'e un sondage t\'el\'ephonique aupr\`es de 30 de nos clients. Les r\'esultats sont les suivants : 

\begin{center}
\begin{tikzpicture}[scale=2.5]

\newcounter{a}
\newcounter{b}
\foreach \p/\t in {8/N'ont rien re\c{c}u, 6/Ont re\c{c}u 2 ou 3 fois, 5/N'ont pas eu envie de lire la lettre~~~~~~~~~~~~~~~~~~~~~~~~~~~~~~~,
                   4/Envisagent de passer \`a l'agence, 7/~~~~~~~~~~~~~~~~~Ne sont pas interr\'ess\'es}
  {
    \setcounter{a}{\value{b}}
    \addtocounter{b}{\p}
    \slice{\thea/30*360}
          {\theb/30*360}
          {\p}
          {\t}
  }

\end{tikzpicture}
\end{center}

\textbf{II. Analyse}

En \'etudiant les r\'esulats, on observe deux points importants : 
\begin{itemize}
\item Une majorit\'e des clients n'ont pas re\c{c}u la lettre ou l'ont re\c{c}u en plusieurs exemplaires
\item Les clients l'ayant re\c{c}u n'ont pas \'et\'e int\'erress\'es par notre offre.
\end{itemize}
\hfill \\

En observant l'annuaire utilis\'e dans le cadre du publipostage des lettres, nous avons en effet observ\'e une inconsistance dans le formattage des donn\'es. Ces inconstances, bien qu'anodines au premier abord, s'av\`erent en r\'ealit\'e \^{e}tre la source du probl\`eme d'addressage.

\hfill \\    
\textbf{III. Mesures correctives}

Pour palier ces probl\`emes, nous avons proc\'ed\'e \`a deux mesures correctives : 

Tout d'abord nous avons enti\`erement revu le formattage de notre anuaire, et mis en place des r\`egles de mise en forme automatique. L'interface qui permet de renseigner les informations client a \'egalement \'et\'e revu pour que les erreurs de saisie ne soient pas permises.

En ce qui concerne l'offre promotionnelle, nous avons d\'ecid\'e de revoir la lettre de mani\`ere \`a mieux cibler notre clientelle. Le nouveau design sera disponible d'ici tr\`es peu de temps.


\hfill \\

Nous sommes convaincus que la prochaine action de mailing, qui sera r\'ealis\'ee \`a l'approche des prochaines vacances, ne sera donc pas soumise aux m\^{e}mes probl\`emes que celle-ci.
\closing{}
\end{letter}
\end{document}